\documentclass[11pt, oneside]{article}   	% use "amsart" instead of "article" for AMSLaTeX format
\usepackage{geometry}                		% See geometry.pdf to learn the layout options. There are lots.
\geometry{letterpaper}                   		% ... or a4paper or a5paper or ... 
%\geometry{landscape}                		% Activate for rotated page geometry
%\usepackage[parfill]{parskip}    		% Activate to begin paragraphs with an empty line rather than an indent
\usepackage{graphicx}				% Use pdf, png, jpg, or eps§ with pdflatex; use eps in DVI mode
								% TeX will automatically convert eps --> pdf in pdflatex		
\usepackage{amssymb}

%SetFonts

%SetFonts


\title{Selection for the $\omega$-Meson Topology from Pythia Analysis}
\author{Michael H. Wood}
%\date{}							% Activate to display a given date or no date

\begin{document}
\maketitle

\begin{table}[htp]
\caption{Listing of the particle selection from the CLAS6 $\omega$-meson analysis of Pythia simulation.}
\begin{center}
\begin{tabular}{| p{2.5cm} | p{2.5cm} | p{2.5cm} | p{2.5cm} |}
\hline
N($\pi^{+}$)$\ge$1, N($\pi^{-}$)$\ge$1, N($\pi^{0}$)$\ge$1 & 
N($\pi^{+}$)=1, N($\pi^{-}$)=1, N($\pi^{0}$)=1 & 
N($\pi^{+}$)$\ge$1, N($\pi^{-}$)$\ge$1, N($\gamma$)$\ge$2 & 
N($\pi^{+}$)=1, N($\pi^{-}$)=1, N($\gamma$)=2 \\  \hline \hline

N($\pi^{+}$)$\ge$1, N($\pi^{-}$)$\ge$1, N($\pi^{0}$)$\ge$1, N($\omega$)$\ge$1 & 
N($\pi^{+}$)=1, N($\pi^{-}$)=1, N($\pi^{0}$)=1, N($\omega$)=1 & 
N($\pi^{+}$)$\ge$1, N($\pi^{-}$)$\ge$1, N($\gamma$)$\ge$2, N($\omega$)$\ge$1 & 
N($\pi^{+}$)=1, N($\pi^{-}$)=1, N($\gamma$)=2, N($\omega$)=1 \\  \hline \hline

N($\pi^{+}$)$\ge$1, N($\pi^{-}$)$\ge$1, N($\pi^{0}$)$\ge$1, N($\omega$)=0 & 
N($\pi^{+}$)=1, N($\pi^{-}$)=1, N($\pi^{0}$)=1, N($\omega$)=0 & 
N($\pi^{+}$)$\ge$1, N($\pi^{-}$)$\ge$1, N($\gamma$)$\ge$2, N($\omega$)=0 & 
N($\pi^{+}$)=1, N($\pi^{-}$)=1, N($\gamma$)=2, N($\omega$)=0 \\  \hline \hline

N($\pi^{+}$)$\ge$1, N($\pi^{-}$)$\ge$1, N($\pi^{0}$)$\ge$1, N($\omega$)=0, N($\eta$)=0, N($\eta^{\prime}$)=0 & 
N($\pi^{+}$)=1, N($\pi^{-}$)=1, N($\pi^{0}$)=1, N($\omega$)=0, N($\eta$)=0, N($\eta^{\prime}$)=0 & 
N($\pi^{+}$)$\ge$1, N($\pi^{-}$)$\ge$1, N($\gamma$)$\ge$2, N($\omega$)=0, N($\eta$)=0, N($\eta^{\prime}$)=0 & 
N($\pi^{+}$)=1, N($\pi^{-}$)=1, N($\gamma$)=2, N($\omega$)=0 N($\eta$)=0, N($\eta^{\prime}$)=0 \\  \hline

\hline
\end{tabular}
\end{center}
\label{default}
\end{table}
%

%\section{}
%\subsection{}



\end{document}  